%%%%%%%%%%%%%%%%%%%%%%%%%%%%%%%%%%%%%%%%%
% Journal Article
% LaTeX Template
% Version 1.4 (15/5/16)
%
% This template has been downloaded from:
% http://www.LaTeXTemplates.com
%
% Original author:
% Frits Wenneker (http://www.howtotex.com) with extensive modifications by
% Vel (vel@LaTeXTemplates.com)
%
% License:
% CC BY-NC-SA 3.0 (http://creativecommons.org/licenses/by-nc-sa/3.0/)
%
%%%%%%%%%%%%%%%%%%%%%%%%%%%%%%%%%%%%%%%%%

%----------------------------------------------------------------------------------------
%	PACKAGES AND OTHER DOCUMENT CONFIGURATIONS
%----------------------------------------------------------------------------------------

\documentclass[twoside,twocolumn]{article}

\usepackage{blindtext} % Package to generate dummy text throughout this template 

\usepackage[sc]{mathpazo} % Use the Palatino font
\usepackage[T1]{fontenc} % Use 8-bit encoding that has 256 glyphs
\linespread{1.05} % Line spacing - Palatino needs more space between lines
\usepackage{microtype} % Slightly tweak font spacing for aesthetics

\usepackage[english]{babel} % Language hyphenation and typographical rules

\usepackage[hmarginratio=1:1,top=32mm,columnsep=20pt]{geometry} % Document margins
\usepackage[hang, small,labelfont=bf,up,textfont=it,up]{caption} % Custom captions under/above floats in tables or figures
\usepackage{booktabs} % Horizontal rules in tables

\usepackage{lettrine} % The lettrine is the first enlarged letter at the beginning of the text

\usepackage{enumitem} % Customized lists
\setlist[itemize]{noitemsep} % Make itemize lists more compact

\usepackage{abstract} % Allows abstract customization
\renewcommand{\abstractnamefont}{\normalfont\bfseries} % Set the "Abstract" text to bold
\renewcommand{\abstracttextfont}{\normalfont\small\itshape} % Set the abstract itself to small italic text

\usepackage{titlesec} % Allows customization of titles
\renewcommand\thesection{\Roman{section}} % Roman numerals for the sections
\renewcommand\thesubsection{\roman{subsection}} % roman numerals for subsections
\titleformat{\section}[block]{\large\scshape\centering}{\thesection.}{1em}{} % Change the look of the section titles
\titleformat{\subsection}[block]{\large}{\thesubsection.}{1em}{} % Change the look of the section titles

\usepackage{fancyhdr} % Headers and footers
\pagestyle{fancy} % All pages have headers and footers
\fancyhead{} % Blank out the default header
\fancyfoot{} % Blank out the default footer
\fancyhead[C]{Running title $\bullet$ May 2016 $\bullet$ Vol. XXI, No. 1} % Custom header text
\fancyfoot[RO,LE]{\thepage} % Custom footer text

\usepackage{titling} % Customizing the title section

\usepackage{hyperref} % For hyperlinks in the PDF
\usepackage{amsmath}
\usepackage{amssymb}

%----------------------------------------------------------------------------------------
%	TITLE SECTION
%----------------------------------------------------------------------------------------

\setlength{\droptitle}{-4\baselineskip} % Move the title up

\pretitle{\begin{center}\Huge\bfseries} % Article title formatting
\posttitle{\end{center}} % Article title closing formatting
\title{This is a template Article - lets fill in on the way as we finish parts} % Article title
\author{%
\textsc{[your names fix later], Ina k. B. Kullmann}\thanks{A thank you or further information - fill in} \\[1ex] % Your name
\normalsize University of Oslo \\ % Your institution
\normalsize \href{mailto:i.k.b.kullmann@fys.uio.no}{i.k.b.kullmann@fys.uio.no} % Your email address
\and % Uncomment if 2 authors are required, duplicate these 4 lines if more
\textsc{Jane Smith}\thanks{Corresponding author} \\[1ex] % Second author's name
\normalsize University of Utah \\ % Second author's institution
\normalsize \href{mailto:jane@smith.com}{jane@smith.com} % Second author's email address
}
\date{\today} % Leave empty to omit a date
\renewcommand{\maketitlehookd}{%
\begin{abstract}
\noindent \blindtext % Dummy abstract text - replace \blindtext with your abstract text
\end{abstract}
}

%----------------------------------------------------------------------------------------

\begin{document}

% Print the title
\maketitle

%----------------------------------------------------------------------------------------
%	ARTICLE CONTENTS
%----------------------------------------------------------------------------------------

\section{Introduction}

\lettrine[nindent=0em,lines=3]{L} orem ipsum dolor sit amet, consectetur adipiscing elit.
\blindtext % Dummy text

\blindtext % Dummy text

%------------------------------------------------

\section{part 1 - rename sections later, do not need the numbering of the exercises}
\subsection{ Exe. 1a}

\textbf{Show that the unperturbed Hamiltonian $\hat{H}_0$ and $\hat{V}$ commute both with $\hat{S}_z$ and $\hat{S}^2$. }

\begin{align*}
\hat{H}_0 = \Sigma_{p\sigma} (p-1) a_{p\sigma}^\dagger a_{p\sigma}
\hat{S}_z = \frac{1}{2} \Sigma_{pq} \sigma a_{p\sigma}^\dagger a_{p\sigma}
\end{align*}

first rewriting the products: 

\begin{align*}
\hat{S}_z \hat{H}_0 &= \frac{1}{2} \Sigma_{p\sigma} \sigma a_{p\sigma}^\dagger a_{p\sigma} \cdot  \Sigma_{p\sigma} (p-1) a_{p\sigma}^\dagger a_{p\sigma} \\
&= \frac{1}{2} \Sigma_{p\sigma}  \Sigma_{qb} \sigma (q-1) a_{p\sigma}^\dagger a_{p\sigma}  a_{qb}^\dagger a_{qb} \\
&= \frac{1}{2} \Sigma_{p\sigma}  \Sigma_{qb} \sigma (q-1) a_\alpha^\dagger a_\alpha  a_\beta^\dagger a_\beta \\
\hat{H}_0 \hat{S}_z  &= \Sigma_{p\sigma} (p-1) a_{p\sigma}^\dagger a_{p\sigma} \cdot  \frac{1}{2} \Sigma_{p\sigma} \sigma a_{p\sigma}^\dagger a_{p\sigma} \\
&= \frac{1}{2} \Sigma_{p\sigma}  \Sigma_{qb} \sigma (q-1) a_{qb}^\dagger a_{qb} a_{p\sigma}^\dagger a_{p\sigma} \\
&= \frac{1}{2} \Sigma_{p\sigma}  \Sigma_{qb} \sigma (q-1) a_\beta^\dagger a_\beta a_\alpha^\dagger a_\alpha \\
\end{align*}

so that the commutation relation becomes:

\begin{align*}
[\hat{S}_z, \hat{H}_0] &= \hat{S}_z \hat{H}_0 - \hat{H}_0  \hat{S}_z \\
&= \frac{1}{2} \Sigma_{p\sigma}  \Sigma_{qb} \sigma (q-1) a_\alpha^\dagger a_\alpha  a_\beta^\dagger a_\beta - \frac{1}{2} \Sigma_{p\sigma}  \Sigma_{qb} \sigma (q-1) a_\beta^\dagger a_\beta a_\alpha^\dagger a_\alpha \\
&= \frac{1}{2} \Sigma_{p\sigma}  \Sigma_{qb} \sigma (q-1) \big( a_\alpha^\dagger a_\alpha  a_\beta^\dagger a_\beta - a_\beta^\dagger a_\beta a_\alpha^\dagger a_\alpha \big) \\
\end{align*}

which can only be zero if 


\section{Methods}

Maecenas sed ultricies felis. Sed imperdiet dictum arcu a egestas. 
\begin{itemize}
\item Donec dolor arcu, rutrum id molestie in, viverra sed diam
\item Curabitur feugiat
\item turpis sed auctor facilisis
\item arcu eros accumsan lorem, at posuere mi diam sit amet tortor
\item Fusce fermentum, mi sit amet euismod rutrum
\item sem lorem molestie diam, iaculis aliquet sapien tortor non nisi
\item Pellentesque bibendum pretium aliquet
\end{itemize}
\blindtext % Dummy text

Text requiring further explanation\footnote{Example footnote}.

%------------------------------------------------

\section{Results}

\begin{table}
\caption{Example table}
\centering
\begin{tabular}{llr}
\toprule
\multicolumn{2}{c}{Name} \\
\cmidrule(r){1-2}
First name & Last Name & Grade \\
\midrule
John & Doe & $7.5$ \\
Richard & Miles & $2$ \\
\bottomrule
\end{tabular}
\end{table}

\blindtext % Dummy text

\begin{equation}
\label{eq:emc}
e = mc^2
\end{equation}

\blindtext % Dummy text

%------------------------------------------------

\section{Discussion}

\subsection{Subsection One}

A statement requiring citation \cite{Figueredo:2009dg}.
\blindtext % Dummy text

\subsection{Subsection Two}

\blindtext % Dummy text

%----------------------------------------------------------------------------------------
%	REFERENCE LIST
%----------------------------------------------------------------------------------------

\begin{thebibliography}{99} % Bibliography - this is intentionally simple in this template

\bibitem[Figueredo and Wolf, 2009]{Figueredo:2009dg}
Figueredo, A.~J. and Wolf, P. S.~A. (2009).
\newblock Assortative pairing and life history strategy - a cross-cultural
  study.
\newblock {\em Human Nature}, 20:317--330.
 
\end{thebibliography}

%----------------------------------------------------------------------------------------

\end{document}
